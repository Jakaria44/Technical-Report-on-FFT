\documentclass{report}
\usepackage{graphicx} % Required for inserting images
\usepackage[a4paper, total={6in, 9in}]{geometry}
\usepackage{amsmath}
\usepackage{enumitem}
\usepackage{minted}
\usepackage{hyperref}
\hypersetup{
    colorlinks=true,
    linkcolor=blue,
    filecolor=magenta,      
    urlcolor=cyan,
    pdftitle={Fast Fourier transform}
    }

\begin{document}

\tableofcontents

\newpage
\chapter{Implementation}
Here we will discuss the implementation of the Fast Fourier Transform algorithm in C++.

\section{Complex number}

\begin{minted}[frame=lines, linenos, fontsize=\large]
{c++}
struct Complex {
	double real, imag;
	Complex() : real(0), imag(0) {}
	Complex(double real, double imag) : real(real), imag(imag) {}
	Complex operator+(const Complex &other) const {
		return Complex(real + other.real, imag + other.imag);
	}
	Complex operator-(const Complex &other) const {
		return Complex(real - other.real, imag - other.imag);
	}
	Complex operator*(const Complex &other) const {
		return Complex(
			real * other.real - imag * other.imag,
			real * other.imag + imag * other.real
		);
	}
};
\end{minted}

\section{Fast Fourier Transform}

\begin{minted}[frame=lines, linenos, fontsize=\large]
{c++}
void fft(vector<Complex> &a, bool invert) {
	int n = a.size();
	if (n == 1) return;
	vector<Complex> a0(n / 2), a1(n / 2);
	for (int i = 0; 2 * i < n; i++) {
		a0[i] = a[2 * i];
		a1[i] = a[2 * i + 1];
	}
	fft(a0, invert);
	fft(a1, invert);
	double ang = 2 * M_PI / n * (invert ? -1 : 1);
	Complex w(1), wn(cos(ang), sin(ang));
	for (int i = 0; 2 * i < n; i++) {
		a[i] = a0[i] + w * a1[i];
		a[i + n / 2] = a0[i] - w * a1[i];
		if (invert) {
			a[i] = a[i] * 0.5;
			a[i + n / 2] = a[i + n / 2] * 0.5;
		}
		w = w * wn;
	}
}

\end{minted}

\section{Multiplication of two polynomials}
\begin{minted}[frame=lines, linenos, fontsize=\large]
{c++}
vector<int> multiply(vector<int> &a, vector<int> &b) {
	vector<Complex> fa(a.begin(), a.end()), 
                        fb(b.begin(), b.end());
	int n = 1;
	while (n < max(a.size(), b.size())) n <<= 1;
	n <<= 1;
	fa.resize(n);
	fb.resize(n);
	fft(fa, false);
	fft(fb, false);
	for (int i = 0; i < n; i++) fa[i] = fa[i] * fb[i];
	fft(fa, true);
	vector<int> result(n);
	for (int i = 0; i < n; i++) 
            result[i] = int(fa[i].real + 0.5);
	return result;
}

\end{minted}






\end{document}